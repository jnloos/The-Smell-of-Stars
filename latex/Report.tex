\documentclass[a4paper, 12pt]{article}

% Sprache
% \usepackage[ngerman]{babel}
\usepackage[english]{babel}

% Pakete
\usepackage[utf8]{inputenc}
\usepackage[T1]{fontenc}
\usepackage{amsmath}
\usepackage{amssymb}
\usepackage{geometry}
\usepackage{fancyhdr}
\usepackage{graphicx}
\usepackage{hyperref}

\setlength{\parindent}{0pt}
\geometry{a4paper, margin=1in, headsep=35pt}

% Referenzen
\usepackage{csquotes}
\usepackage[style=verbose-ibid,backend=bibtex]{biblatex}

% Kopfzeile anpassen
\pagestyle{fancy}
\fancyhead{}
\fancyhead[L]{\veranstaltung \\ Projekt: \blatt}
\fancyhead[C]{\textbf{Jan-Niclas Loosen}\\ \textbf{Mat.Nr. 1540907}}
\fancyhead[R]{Universität Trier\\ \today}

% Variablen für Anpassungen
\newcommand{\blatt}{Report} % Nummer des Übungsblatts
\newcommand{\veranstaltung}{Softwaretechnik II} % Name der Veranstaltung
% \bibliography{xyz} % Name der BibTex-Datei

\begin{document}
	
	\subsection*{The Smell of Stars – A small study on the correlation between GitHub popularity and code quality.}
	
	\subsection*{Introduction}
	
	\subsection*{Study Design}
	
	\begin{itemize}
		\item Normierung der Ergebnisse gemäß Lines of Code. Dadurch bessere Vergleichbarkeit und Auswertung mittels Cohens-D sowie Mann-Whitney-U Test. 
		\item Anzahl der Sterne war schwierig zu justieren: 10.000 zu viele für PHP, 0 Sterne haben einige leere Projekte mit berücksichtigt. Deshalb finale Entscheidung für 40­--400 Sterne bzw. >4000 Sterne für beliebt.
		\item Keine Berücksichtigung von Sprachen die kompiliert werden müssen
		\item Die Projekte wurden nicht zufällig gezogen. Stattdessen wurde nach Best Match (https://docs.github.com/en/rest/search/search) gesucht. Ursprünglich 120 beliebte und standard Repos pro Sprache -- wegen SonarQube Timeout und Arbeitsspeicherverbrauch teilweise abgestürzt. 
		\item Crawling und insbesondere Scan hat lange gedauert...
	\end{itemize}	
	
	\subsection*{Evaluation}
		\begin{itemize}
		\item Grundsätzlich kein signifikanter Unterschied an Smells und Komplexität
		\item Von den Sprachen abhängig? Python und PHP vs. Ruby und JS
	\end{itemize}	
	
	\subsection*{Conclusion}
	
\end{document}
